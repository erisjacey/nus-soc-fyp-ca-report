\chapter{Introduction}
\label{introduction}

\section{Background}

\subsubsection{Automated Program Repair (APR)}

APR can be described as a suite of technologies whose main aim is to automatically
fix errors or vulnerabilities in software systems~\cite{le2021automatic}.
It hones in on a class of techniques that, at its core, takes a program as input
along with information about its error (e.g. failing tests).
It then produces a patch at the code-level for that program's source to fix its
errors, ideally while maintaining its other correct processes.

At its present, APR has a myraid of practical use cases~\cite{goues2019automated},
four of which include:

\begin{enumerate}
    \item Fixing bugs throughout development
    \item Repairing security vulnerabilities
    \item Intelligent tutoring
    \item Self-healing of performance bottlenecks
\end{enumerate}

\subsubsection{Intelligent Tutoring System (ITS)}

In this project, we are interested in APR's utility in the use case of intelligent tutoring,
specifically for students taking CS-1 programming classes.
The high level idea of APR in this context involves taking in a buggy student submission
and a correct reference solution as inputs, then producing a fixed version of the student's
program as output.

This project is being built on the backbone of the ITS we have been developing.
Some features of this system include leveraging APR techniques, providing students with
dynamic hints and feedback, as well as a robust auto-grader.

While the ITS has slowly been increasingly integrated in various introductory
programming courses (such as CS1010 Programming Methodology), much work can still be done
to improve its performance.
Two such areas include the use of refactoring to increase match rate between reference
solution and incorrect solution, as well as the use of large language models such as Codex.

\subsubsection{The Role of Refactoring in APR}

Refactoring-based program repair is one possible way to employ APR in programming
assignments~\cite{hu2019re}.
The first step to this approach involves refactoring all available correct solutions to
semantically equivalent solutions.
We then match a given incorrect program with the refactored program with the closest match
based on its control flow structure.
Next, we infer the input-output specification of the incorrect program's basic blocks
from the executions of the corect program's aligned basic blocks.
We finally use these specifications to patch the blocks of the incorrect program through
search-based synthesis.

\subsubsection{Codex}
\label{background:codex}

Codex-e has been shown to give the best results among APR strategies, including TBar and Recorder
\cite{fan2022improving}.
Codex-e, like existing APR tools, typically follow a framework involving alignment and repair.
They first search for a structure and variable alignment relation between the incorrect solution
and reference solution, and then find patches that make each aligned variable have the same
semantic behaviour in both fixed and reference solutions.

\begin{multicols}{2}
\begin{lstlisting}[
    language=Python,
    caption={Incorrect solution.},
    label={lst:codex:incorrect-sol}
]
def compute_area(a, b, c):
    sum = (a + b + c)
    s = sum / 0.5
    s = s*(s-a)*(s-b)*(s-c)
    return Math.sqrt(s)
\end{lstlisting}

\columnbreak

\begin{lstlisting}[
    language=Python,
    caption={Reference solution.},
    label={lst:codex:reference-sol}
]
def compute_area(a, b, c):
    s = (a + b + c)
    return Math.sqrt(s*(s-a
        )*(s-b)*(s-c))
\end{lstlisting}
\end{multicols}

\begin{multicols}{2}
\begin{lstlisting}[
    language=Python,
    caption={Correct patch.},
    label={lst:codex:correct-patch}
]
def compute_area(a, b, c):
-   sum = (a + b + c)
-   s = sum / 0.5
-   s = s*(s-a)*(s-b)*(s-c)
-   return Math.sqrt(s)
+   s = (a + b + c)
+   return Math.sqrt(s*(s-a
+       )*(s-b)*(s-c))
\end{lstlisting}

\columnbreak

\begin{lstlisting}[
    language=Python,
    caption={Ideal patch.},
    label={lst:codex:ideal-patch}
]
def compute_area(a, b, c):
    sum = (a + b + c)
-   s = sum / 0.5
+   s = sum * 0.5
    s = s*(s-a)*(s-b)*(s-c)
    return Math.sqrt(s)
\end{lstlisting}
\end{multicols}

Consider the example above, where the problem entails computing the area of a triangle.

Existing APR tools would take in an incorrect solution (Listing~\ref{lst:codex:incorrect-sol})
and a correct reference solution (Listing~\ref{lst:codex:reference-sol}).
The ideal fix to the incorrect solution would be to modify line 3, by changing the division
operator to the multiplication operator (Listing~\ref{lst:codex:ideal-patch}).
Structural alignment is simple because both solutions only have basic block each.
The result of variable alignment is ${a -> a, b -> b, c -> c, s -> s, sum -> null}$.
During the patching process, however, such APR tools will generate a patch that:
\begin{enumerate}
    \item Omits the variable \texttt{sum}, due to its absence in the reference solution.
    \item Replaces expressions involving the variable \texttt{s} with the reference
          solution, due to a mismatch in the semantic behaviour of \texttt{s}.
    \item Replaces expressions of \texttt{return}, due to variable \texttt{s} being
          changed.
\end{enumerate}

We notice that the correct patch generated (Listing~\ref{lst:codex:correct-patch}) may
not be as effective as the ideal patch for learning students.
It might be prudent for us to instead provide personalised feedback by transforming a
reference solution into a "personalised" reference solution.
Using large language models, this new solution will be syntactically similar to the
incorrect solution but also conforms to the semantics of the programming assignment.

\section{Project Objectives}

One of the preliminary steps in the ITS framework requires the refactoring of reference correct
solutions in the hopes of more closely matching incorrect student programs.
As such, the first part of this project involves the implementation of refactoring rules as
outlined by an APR-centered paper~\cite{hu2019re}.
These rules, implemented in the ITS backend, are intended to improve the robustness of the
application.
The ITS will be able to increase the likelihood of finding a match between a given incorrect
student solution and reference correct solution to be able to kickstart the repair process.

Subsequently, the second part revolves around investigating large language models (LLMs),
specifically Codex, and their efficacy in the APR process for CS-1 programming assignments.
More precisely, we aim to develop a specialised APR process that minimises the "syntactic
distance" between incorrect and reference solutions while preserving the semantic
correctness of the reference solution.

Through this research portion of the project, we aim to make use of Codex in the most
effective way possible for our use case, which is to increase the efficiency of both
teachers' grading performance and students' learning ease.

We realise the following challenges to consider:
\begin{itemize}
    \item How do we ensure the semantic correctness of reference solution during the program
          transformation of the language model?
    \item Can we perform such program transformations without any transformation examples?
          If not, how should examples be leveraged?
    \item How do we encode the "syntactic distance" information to guide the language model?
\end{itemize}

We finally arrive at a few actionable items worth investigating:
\begin{itemize}
    \item Can existing program transformation tools help refactor reference solutions
          (e.g. FixMorph, SYDIT, Refazer)?
    \item How severe the is the problem regarding inaccurate reference solutions?
          (i.e. How many incorrect solutions are fixed with larger patches, compared
           to minimal patches?)
    \item How does Codex perform in fixing incorrect CS-1 solutions \textbf{without reference solutions}?
    \item How does Codex perform in fixing incorrect CS-1 solutions \textbf{with a single reference
          solution} (without program transformation)?
\end{itemize}

% TEMPLATE BELOW %%%%%%%%%%%%%%%%%%%%%%%%%%%%%%%%%%%%%%%%%%%%%%%%%%%%%%%%%%%%%%%%%%%%%%%%%%%%%%%%%%%

%\section{Contributions}
%This is an example of an enumeration.
%
%\begin{enumerate}
%    \item Point one.
%    \item Point two.
%    \item Point three.
%\end{enumerate}
%
%\begin{lstlisting}[
%    language=TeX,
%    caption={Example Listing.}
%]
%\title{An Example HYP Final Report}
%\author{Harold Finch}
%\projyear{2015}
%\projnumber{123-45-6789}
%\advisor{Dr. Samaritan}
%\deliverables{
%    \item Report: 1 Volume
%}
%\maketitle
%\end{lstlisting}