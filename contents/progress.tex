\chapter{Progress Made So Far}
We have made significant progress in the two areas mentioned in Chapter 1 (Introduction).

\section{Implementation of Refactoring Rules}

This reference paper~\cite{hu2019re} features a set of 16 refactoring rules divided into 5 categories.
We have implemented them in the back-end
\texttt{nus-se/its-baseline} repository,
which has been developed in Java.

The high-level design of the subcomponent responsible for this (aptly named \texttt{its-refactoring})
involves each refactoring rule (implemented as a Java sub-class of the \texttt{RefactoringRule} interface)
taking in a program in the form of the ITS' version of an abstract syntax tree, and return the refactored
program as output, provided refactoring can be done.

A summary of the rules and their respective implementing Java classes can be found
\href{https://docs.google.com/document/d/1MnEVEUaiQ1A2MWca_qPiktKZdP6hV-1CLTXcWwK6PyE/edit}{here}.

An example highlighting the semantics refactoring rule A1(a): successor statements to a
conditional jump can be found in the next page:

\pagebreak

\begin{multicols}{2}
\begin{lstlisting}[
    language=Python,
    caption={Code before refactoring.}
]
def example(cond):
    if (cond):
        return 0
    return 1
\end{lstlisting}

\columnbreak

\begin{lstlisting}[
    language=Python,
    caption={Code after refactoring.}
]
def example(cond):
    if (cond):
        return 0
    else:
        return 1
\end{lstlisting}
\end{multicols}

\section{Codex and APR in CS-1 Assignments}

\subsection{Preliminary Investigation}

\subsection{Investigation Alternatives}
