\chapter{Progress Made So Far}
We have made significant progress in the two areas mentioned in Chapter 1 (Introduction).

\section{Implementation of Refactoring Rules}

This reference paper~\cite{hu2019re} features a set of 16 refactoring rules divided into 5 categories.
We have implemented them in the back-end
\texttt{nus-se/its-baseline} repository,
which has been developed in Java.

The high-level design of the subcomponent responsible for this (aptly named \texttt{its-refactoring})
involves each refactoring rule (implemented as a Java sub-class of the \texttt{RefactoringRule} interface)
taking in a program in the form of the ITS' version of an abstract syntax tree, and returning the
refactored program as output, provided refactoring can be done.

A summary of the rules and their respective implementing Java classes can be found
\href{https://docs.google.com/document/d/1MnEVEUaiQ1A2MWca_qPiktKZdP6hV-1CLTXcWwK6PyE/edit}{here}.

An example highlighting the semantics refactoring rule A1(a): successor statements to a
conditional jump can be found in the next page:

\pagebreak

\begin{multicols}{2}
\begin{lstlisting}[
    language=Python,
    caption={Code \textbf{before} refactoring.}
]
def example(cond):
    if (cond):
        return 0
    return 1
\end{lstlisting}

\columnbreak

\begin{lstlisting}[
    language=Python,
    caption={Code \textbf{after} refactoring.}
]
def example(cond):
    if (cond):
        return 0
    else:
        return 1
\end{lstlisting}
\end{multicols}

\section{Codex and APR in CS-1 Assignments}

Codex-e has been shown to give the best results among APR strategies, including TBar and Recorder
~\cite{fan2022improving}.
We have thus decided to investigate more into the utility of
\href{https://openai.com/blog/openai-codex/}{this APR tool},
and specifically on whether or not we can leverage it in our use case, which would be the repair
of student submissions for CS-1 assignments.

\subsection{Preliminary Investigation}

Codex-e, like existing APR tools, typically follow a framework involving alignment and repair.
They first search for a structure and variable alignment relation between the incorrect solution
and reference solution and find patches that make each aligned variable have the same semantic
behaviour in both fixed and reference solutions.

\pagebreak

\begin{multicols}{2}
\begin{lstlisting}[
    language=Python,
    caption={Incorrect solution.}
]
def compute_area(a, b, c):
    sum = (a + b + c)
    s = sum / 0.5
    s = s*(s-a)*(s-b)*(s-c)
    return Math.sqrt(s)
\end{lstlisting}

\columnbreak

\begin{lstlisting}[
    language=Python,
    caption={Correct solution.}
]
def compute_area(a, b, c):
    s = (a + b + c)
    return Math.sqrt(s*(s-a
        )*(s-b)*(s-c))
\end{lstlisting}
\end{multicols}

We realise the following challenges to consider:
\begin{itemize}
    \item How do we ensure the semantic correctness of reference solution during the program
          transformation of the language model?
    \item Can we perform such program transformations without any transformation examples?
          If not, how should examples be leveraged?
    \item How do we encode the "syntactic distance" information to guide the language model?
\end{itemize}

We finally arrive at a few actionable items worth investigating:
\begin{itemize}
    \item Can existing program transformation tools help refactor reference solutions
          (e.g. FixMorph, SYDIT, Refazer)?
    \item How severe the is the problem regarding inaccurate reference solutions?
          (i.e. How many incorrect solutions are fixed with larger patches, compared
           to minimal patches?)
    \item How does Codex perform in fixing incorrect CS-1 solutions \textbf{without reference solutions}?
    \item How does Codex perform in fixing incorrect CS-1 solutions \textbf{with a single reference
          solution} (without program transformation)?
\end{itemize}

\subsection{Investigation Findings}

We have developed a Python framework to investigate the efficacy and accuracy of using Codex in
CS-1 assignments.
Through this framework, we have made significant progress in addressing our actionable items regarding
fixing incorrect CS-1 solutions (with and without a reference solution), as well as in tackling our
challenge regarding encoding the "syntactic distance" to hopefully better guide the language model.

\subsubsection{Generating code patches for incorrect student submissions}

Given a single reference solution and incorrect solution, our framework is able to generate a patch
solution using the Codex \texttt{edit} API. We are also able to run this patching using multiple
parameters allowed by the API, including \texttt{n} (number of patches per submission) and
\texttt{temperature}.
For our preliminary analyses, we have used the parameters \texttt{n = 1, 2, 5} and
\texttt{temperature = 0.8}.

Upon the generation of the patches for each student submission, we are also able to analyse the
semantic correctness of each patch solution, by subjecting it to the test suite of its corresponding
problem.
We use the following metrics for analysis:
\begin{itemize}
    \item \texttt{pass@k}.
    \item Pass rate of all submissions for each problem.
          For this metric, we say that a submission is considered "passed" as long as any of its
          patches manage to pass all the test cases to which it is subjected.
\end{itemize}

The following figure shows the results of our findings with our specified parameters:

\textit{INSERT FIGURE HERE}

\subsubsection{Generating code completions given a problem description}

Our framework is also able to generate a variable number of completion solutions, given a natural
language prompt in the form of a problem description (see Appendix A for examples).
Similar to above, we are able to vary \texttt{n} (number of completions per problem]) and
\texttt{temperature}.
For our preliminary analyses, we have used the parameters \texttt{n = 5} and
\texttt{temperature = 0.8}.

% TODO results?

\subsubsection{Devising a heuristic for measuring "syntactic distance" between two solutions}

Finally, we have managed to design a heuristic to compute the "syntactic distance" between two solutions
written in C.
As of now, we are only able to perform this heuristic on a pair of reference solution and its corresponding
(generated) patch solution.
Moreover, we only perform this heuristic if the patch solution can be compiled and is tested to be
semantically correct.
Analysing an incorrect patch solution in this step is simply meaningless and/or impossible.

The execution flow of this heuristic is as follows:
\begin{enumerate}
    \item We first parse each solution, written in C, using
          \href{https://github.com/eliben/pycparser}{pycparser, a third-party library written in Python
                to parse C code}.
    \item We then obtain each parsed solution as a custom \texttt{c\_ast.Node} object.
    \item For each \texttt{c\_ast.Node} object, we convert it into a \texttt{zss.Node} object.
          This is a necessary step as:
    \begin{itemize}
        \item It allows the converted object to now be processed in the final step of our heuristic.
        \item Each object will be assigned its corresponding label, which we use as the main source
              of "syntactic difference" across code solutions. (i.e. nodes with different labels
              will be considered to have a "syntactic distance" of \texttt{1})
    \end{itemize}
    \item Finally, to obtain the distance, we perform the algorithm to do so, designed by
          Zhang and Shasha~\cite{zhang1989simple}, and whose example implementations can be found
          \href{https://pythonhosted.org/zss/#module-zss}{here}.
\end{enumerate}