\chapter{Plan For Next Semester}

\section{Implementation of Refactoring Rules}

The refactoring rules that we have implemented in the ITS backend
(see: Section~\ref{sec:implementation-of-refactoring-rules})
are for the most part self-contained within the \texttt{its-refactoring} submodule of
the ITS codebase.
We aim to fully integrate the utilisation of such refactoring rules within the ITS
framework in the upcoming semester.

\section{Codex and APR in CS-1 Assignments}

The low-hanging fruit to continue to work on would be the other challenges and
actionable items highlighted in Section~\ref{sec:project-objectives} of this report.
These include:
\begin{itemize}
    \item How do we ensure the semantic correctness of reference solution during the program
          transformation of the language model?
    \item Can we perform such program transformations without any transformation examples?
          If not, how should examples be leveraged?
    \item Can existing program transformation tools help refactor reference solutions
          (e.g. FixMorph, SYDIT, Refazer)?
    \item How severe the is the problem regarding inaccurate reference solutions?
          (i.e. How many incorrect solutions are fixed with larger patches, compared
           to minimal patches?)
\end{itemize}

Our goal, in a nutshell, is to continue to develop the framework we have initiated
in the hopes of optimising Codex, maximising its repair success rate and minimising
syntactic changes to incorrect solutions.
We hope to make use of this specialised APR technique for our particular use case,
which is in the context of intelligent tutoring.

%\section{Other APR Tools}

