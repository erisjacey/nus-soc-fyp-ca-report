\documentclass[fyp,12pt]{socreport}

% Some generic packets.
\usepackage{color, colortbl}
\usepackage{url}
\usepackage{graphicx}
\usepackage{caption}
\usepackage{subcaption}
\usepackage{pgfplots}
\usepackage{tabularx}
\usepackage{multirow}
\usepackage{multicol}
\usepackage{listings}
\usepackage{fullpage}
\usepackage{hyperref}
\usepackage[numbers]{natbib}
\pgfplotsset{width=10cm,compat=1.9}

% Sets the root path to look for all images.
\graphicspath{{images/}}

% Sets default options for listings.
\renewcommand\lstlistlistingname{List of Listings}
\newcommand*\lstinputpath[1]{\lstset{inputpath=#1}}
\lstinputpath{listings}
\lstset{frame=single, tabsize=2, captionpos=b}
\newcommand{\itab}[1]{\hspace{0em}\rlap{#1}}
\newcommand{\tab}[1]{\hspace{.11\textwidth}\rlap{#1}}

\begin{document}
\pagenumbering{roman}

% Replace as necessary
\title{Introductory Programming Assignment Repair via Program Transformations}
\author{Eris Jacey Masagca}
\projyear{AY 2022/2023}
\projnumber{H049520}
\advisor{Provost's Chair Prof Abhik Roychoudhury}
\deliverables{
    \item \itab{Report:} \tab{1 Volume}
}
\maketitle

% Replace as necessary
\begin{abstract}

The use of automated feedback students attending introductory programming courses has become
increasingly popular due to its efficient and robust approach to feedback generation and error
detection, which can improve the grading performance of teachers and help students learn better
at the same time~\cite{haldeman2021automated}.
One of the greatest challenges in this field includes the accuracy of automated program repair (APR)
for incorrect student programming assignment submissions.
Existing techniques, such as TBar and Recorder, suffer from limitations like needing to rely on
a rich reference solution set for the same assignment.
In our work, we have been working on supporting an intelligent tutoring system (ITS), which
leverages and improves on APR techniques.
The end-goal of this application is to aid students in CS-1 programming classes.
The overall performance of this project will initially be gauged based on various student submissions
and model correct solutions of programming assignments in one of the CS-1 classes at the National
University of Singapore (NUS): CS1010 Programming Methodology.

\begin{descriptors} % TODO
    \item \itab{TODO}	\tab{TODO}
%    \item \itab{C.2.1}	\tab{Network Architecture and Design}
%    \item \itab{C.2.2}	\tab{Network Protocols}
%    \item \itab{C.2.4}	\tab{Distributed Systems}
%    \item \itab{C.4} 	\tab{Performance of Systems}
%    \item \itab{I.2.9}	\tab{Robotics}
\end{descriptors}
\begin{keywords}
    Software engineering \\
    Programming languages \& systems \\
    Automated program repair
\end{keywords}

% Replace/Delete as necessary
\begin{implement}
    \item{Ubuntu 20.04.1 LTS, Windows 10}
    \item{Java 11 SE, IntelliJ IDEA 2020.2.3}
    \item{Python 3.8.5, PyCharm 2021.1.2}
\end{implement}

\end{abstract}

% Replace as necessary
%\begin{acknowledgement}
%See \texttt{README.md} of this repository.
%\end{acknowledgement}

%\listoffigures % Remove if no figures
\listoftables % Remove if no tables
\lstlistoflistings % Remove if no listings
\tableofcontents

% Actual contents in the contents folder.
% Include additional tex files here.
\chapter{Introduction}
\label{introduction}

\section{Background}

\subsubsection{Automated Program Repair (APR)}

APR can be described as a suite of technologies whose main aim is to automatically
fix errors or vulnerabilities in software systems~\cite{le2021automatic}.
It hones in on a class of techniques that, at its core, takes a program as input
along with information about its error (e.g. failing tests).
It then produces a patch at the code-level for that program's source to fix its
errors, ideally while maintaining its other correct processes.

At its present, APR has a myraid of practical use cases~\cite{goues2019automated},
four of which include:

\begin{enumerate}
    \item Fixing bugs throughout development
    \item Repairing security vulnerabilities
    \item Intelligent tutoring
    \item Self-healing of performance bottlenecks
\end{enumerate}

\subsubsection{Intelligent Tutoring System (ITS)}

In this project, we are interested in APR's utility in the use case of intelligent tutoring,
specifically for students taking CS-1 programming classes.
The high level idea of APR in this context involves taking in a buggy student submission
and a correct reference solution as inputs, then producing a fixed version of the student's
program as output.

This project is being built on the backbone of the ITS we have been developing.
Some features of this system include leveraging APR techniques, providing students with
dynamic hints and feedback, as well as a robust auto-grader.

While the ITS has slowly been increasingly integrated in various introductory
programming courses (such as CS1010 Programming Methodology), much work can still be done
to improve its performance.
Two such areas include the use of refactoring to increase match rate between reference
solution and incorrect solution, as well as the use of large language models such as Codex.

\subsubsection{The Role of Refactoring in APR}

Refactoring-based program repair is one possible way to employ APR in programming
assignments~\cite{hu2019re}.
The first step to this approach involves refactoring all available correct solutions to
semantically equivalent solutions.
We then match a given incorrect program with the refactored program with the closest match
based on its control flow structure.
Next, we infer the input-output specification of the incorrect program's basic blocks
from the executions of the corect program's aligned basic blocks.
We finally use these specifications to patch the blocks of the incorrect program through
search-based synthesis.

\subsubsection{Codex}
\label{background:codex}

Codex-e has been shown to give the best results among APR strategies, including TBar and Recorder
\cite{fan2022improving}.
Codex-e, like existing APR tools, typically follow a framework involving alignment and repair.
They first search for a structure and variable alignment relation between the incorrect solution
and reference solution, and then find patches that make each aligned variable have the same
semantic behaviour in both fixed and reference solutions.

\begin{multicols}{2}
\begin{lstlisting}[
    language=Python,
    caption={Incorrect solution.},
    label={lst:codex:incorrect-sol}
]
def compute_area(a, b, c):
    sum = (a + b + c)
    s = sum / 0.5
    s = s*(s-a)*(s-b)*(s-c)
    return Math.sqrt(s)
\end{lstlisting}

\columnbreak

\begin{lstlisting}[
    language=Python,
    caption={Reference solution.},
    label={lst:codex:reference-sol}
]
def compute_area(a, b, c):
    s = (a + b + c)
    return Math.sqrt(s*(s-a
        )*(s-b)*(s-c))
\end{lstlisting}
\end{multicols}

\begin{multicols}{2}
\begin{lstlisting}[
    language=Python,
    caption={Correct patch.},
    label={lst:codex:correct-patch}
]
def compute_area(a, b, c):
-   sum = (a + b + c)
-   s = sum / 0.5
-   s = s*(s-a)*(s-b)*(s-c)
-   return Math.sqrt(s)
+   s = (a + b + c)
+   return Math.sqrt(s*(s-a
+       )*(s-b)*(s-c))
\end{lstlisting}

\columnbreak

\begin{lstlisting}[
    language=Python,
    caption={Ideal patch.},
    label={lst:codex:ideal-patch}
]
def compute_area(a, b, c):
    sum = (a + b + c)
-   s = sum / 0.5
+   s = sum * 0.5
    s = s*(s-a)*(s-b)*(s-c)
    return Math.sqrt(s)
\end{lstlisting}
\end{multicols}

Consider the example above, where the problem entails computing the area of a triangle.

Existing APR tools would take in an incorrect solution (Listing~\ref{lst:codex:incorrect-sol})
and a correct reference solution (Listing~\ref{lst:codex:reference-sol}).
The ideal fix to the incorrect solution would be to modify line 3, by changing the division
operator to the multiplication operator (Listing~\ref{lst:codex:ideal-patch}).
Structural alignment is simple because both solutions only have basic block each.
The result of variable alignment is ${a -> a, b -> b, c -> c, s -> s, sum -> null}$.
During the patching process, however, such APR tools will generate a patch that:
\begin{enumerate}
    \item Omits the variable \texttt{sum}, due to its absence in the reference solution.
    \item Replaces expressions involving the variable \texttt{s} with the reference
          solution, due to a mismatch in the semantic behaviour of \texttt{s}.
    \item Replaces expressions of \texttt{return}, due to variable \texttt{s} being
          changed.
\end{enumerate}

We notice that the correct patch generated (Listing~\ref{lst:codex:correct-patch}) may
not be as effective as the ideal patch for learning students.
It might be prudent for us to instead provide personalised feedback by transforming a
reference solution into a "personalised" reference solution.
Using large language models, this new solution will be syntactically similar to the
incorrect solution but also conforms to the semantics of the programming assignment.

\section{Project Objectives}

One of the preliminary steps in the ITS framework requires the refactoring of reference correct
solutions in the hopes of more closely matching incorrect student programs.
As such, the first part of this project involves the implementation of refactoring rules as
outlined by an APR-centered paper~\cite{hu2019re}.
These rules, implemented in the ITS backend, are intended to improve the robustness of the
application.
The ITS will be able to increase the likelihood of finding a match between a given incorrect
student solution and reference correct solution to be able to kickstart the repair process.

Subsequently, the second part revolves around investigating large language models (LLMs),
specifically Codex, and their efficacy in the APR process for CS-1 programming assignments.
More precisely, we aim to develop a specialised APR process that minimises the "syntactic
distance" between incorrect and reference solutions while preserving the semantic
correctness of the reference solution.

Through this research portion of the project, we aim to make use of Codex in the most
effective way possible for our use case, which is to increase the efficiency of both
teachers' grading performance and students' learning ease.

We realise the following challenges to consider:
\begin{itemize}
    \item How do we ensure the semantic correctness of reference solution during the program
          transformation of the language model?
    \item Can we perform such program transformations without any transformation examples?
          If not, how should examples be leveraged?
    \item How do we encode the "syntactic distance" information to guide the language model?
\end{itemize}

We finally arrive at a few actionable items worth investigating:
\begin{itemize}
    \item Can existing program transformation tools help refactor reference solutions
          (e.g. FixMorph, SYDIT, Refazer)?
    \item How severe the is the problem regarding inaccurate reference solutions?
          (i.e. How many incorrect solutions are fixed with larger patches, compared
           to minimal patches?)
    \item How does Codex perform in fixing incorrect CS-1 solutions \textbf{without reference solutions}?
    \item How does Codex perform in fixing incorrect CS-1 solutions \textbf{with a single reference
          solution} (without program transformation)?
\end{itemize}

% TEMPLATE BELOW %%%%%%%%%%%%%%%%%%%%%%%%%%%%%%%%%%%%%%%%%%%%%%%%%%%%%%%%%%%%%%%%%%%%%%%%%%%%%%%%%%%

%\section{Contributions}
%This is an example of an enumeration.
%
%\begin{enumerate}
%    \item Point one.
%    \item Point two.
%    \item Point three.
%\end{enumerate}
%
%\begin{lstlisting}[
%    language=TeX,
%    caption={Example Listing.}
%]
%\title{An Example HYP Final Report}
%\author{Harold Finch}
%\projyear{2015}
%\projnumber{123-45-6789}
%\advisor{Dr. Samaritan}
%\deliverables{
%    \item Report: 1 Volume
%}
%\maketitle
%\end{lstlisting}
\chapter{Literature Review}

%TO REVIEW (?):
%
%\begin{enumerate}
%    \item~\cite{ahmed2022verifix}
%    \item~\cite{chen2021evaluating}
%    \item~\cite{haldeman2021automated}
%    \item~\cite{hu2019re}
%    \item~\cite{zhang2022repairing}
%\end{enumerate}

\paragraph{Improving automatically generated code from Codex via Automated Program Repair}

\begin{itemize}
    \item Existing APR techniques like TBar and Recoder have their limitations in the context of
          fixing bugs in auto-generated programs.
    \item Codex-e has the potential of outperforming TBar and Recoder in code edit generation.
    \item The conventional way in which APR for introductory programming assignments still relies
          on referencing a rich reference solution set for the same assignment
    \item We propose:
    \begin{itemize}
        \item A matching mechanism to be able to search for similar past submissions from past
              assignments, which can then be used by language models.
        \item An incorrect fix guided refinement mechanism to further improve matching quality
              and repair success rate.
        \item An evaluation of the approach in real student submissions and showing of the value
              of historical assignments in APR
    \end{itemize}
\end{itemize}
\chapter{Progress Made So Far}
We have made significant progress in the two areas mentioned in Chapter 1 (Introduction).

\section{Implementation of Refactoring Rules}

This reference paper~\cite{hu2019re} features a set of 16 refactoring rules divided into 5 categories.
We have implemented them in the back-end
\texttt{nus-se/its-baseline} repository,
which has been developed in Java.

The high-level design of the subcomponent responsible for this (aptly named \texttt{its-refactoring})
involves each refactoring rule (implemented as a Java sub-class of the \texttt{RefactoringRule} interface)
taking in a program in the form of the ITS' version of an abstract syntax tree, and returning the
refactored program as output, provided refactoring can be done.

A summary of the rules and their respective implementing Java classes can be found
\href{https://docs.google.com/document/d/1MnEVEUaiQ1A2MWca_qPiktKZdP6hV-1CLTXcWwK6PyE/edit}{here}.

An example highlighting the semantics refactoring rule A1(a): successor statements to a
conditional jump can be found in the next page:

\pagebreak

\begin{multicols}{2}
\begin{lstlisting}[
    language=Python,
    caption={Code \textbf{before} refactoring.}
]
def example(cond):
    if (cond):
        return 0
    return 1
\end{lstlisting}

\columnbreak

\begin{lstlisting}[
    language=Python,
    caption={Code \textbf{after} refactoring.}
]
def example(cond):
    if (cond):
        return 0
    else:
        return 1
\end{lstlisting}
\end{multicols}

\section{Codex and APR in CS-1 Assignments}

Codex-e has been shown to give the best results among APR strategies, including TBar and Recorder
~\cite{fan2022improving}.
We have thus decided to investigate more into the utility of
\href{https://openai.com/blog/openai-codex/}{this APR tool},
and specifically on whether or not we can leverage it in our use case, which would be the repair
of student submissions for CS-1 assignments.

\subsection{Preliminary Investigation}

Codex-e, like existing APR tools, typically follow a framework involving alignment and repair.
They first search for a structure and variable alignment relation between the incorrect solution
and reference solution and find patches that make each aligned variable have the same semantic
behaviour in both fixed and reference solutions.

\pagebreak

\begin{multicols}{2}
\begin{lstlisting}[
    language=Python,
    caption={Incorrect solution.}
]
def compute_area(a, b, c):
    sum = (a + b + c)
    s = sum / 0.5
    s = s*(s-a)*(s-b)*(s-c)
    return Math.sqrt(s)
\end{lstlisting}

\columnbreak

\begin{lstlisting}[
    language=Python,
    caption={Correct solution.}
]
def compute_area(a, b, c):
    s = (a + b + c)
    return Math.sqrt(s*(s-a
        )*(s-b)*(s-c))
\end{lstlisting}
\end{multicols}

We realise the following challenges to consider:
\begin{itemize}
    \item How do we ensure the semantic correctness of reference solution during the program
          transformation of the language model?
    \item Can we perform such program transformations without any transformation examples?
          If not, how should examples be leveraged?
    \item How do we encode the "syntactic distance" information to guide the language model?
\end{itemize}

We finally arrive at a few actionable items worth investigating:
\begin{itemize}
    \item Can existing program transformation tools help refactor reference solutions
          (e.g. FixMorph, SYDIT, Refazer)?
    \item How severe the is the problem regarding inaccurate reference solutions?
          (i.e. How many incorrect solutions are fixed with larger patches, compared
           to minimal patches?)
    \item How does Codex perform in fixing incorrect CS-1 solutions \textbf{without reference solutions}?
    \item How does Codex perform in fixing incorrect CS-1 solutions \textbf{with a single reference
          solution} (without program transformation)?
\end{itemize}

\subsection{Investigation Findings}

We have developed a Python framework to investigate the efficacy and accuracy of using Codex in
CS-1 assignments.
Through this framework, we have made significant progress in addressing our actionable items regarding
fixing incorrect CS-1 solutions (with and without a reference solution), as well as in tackling our
challenge regarding encoding the "syntactic distance" to hopefully better guide the language model.

\subsubsection{Generating code patches for incorrect student submissions}

Given a single reference solution and incorrect solution, our framework is able to generate a patch
solution using the Codex \texttt{edit} API. We are also able to run this patching using multiple
parameters allowed by the API, including \texttt{n} (number of patches per submission) and
\texttt{temperature}.
For our preliminary analyses, we have used the parameters \texttt{n = 1, 2, 5} and
\texttt{temperature = 0.8}.

Upon the generation of the patches for each student submission, we are also able to analyse the
semantic correctness of each patch solution, by subjecting it to the test suite of its corresponding
problem.
We use the following metrics for analysis:
\begin{itemize}
    \item \texttt{pass@k}.
    \item Pass rate of all submissions for each problem.
          For this metric, we say that a submission is considered "passed" as long as any of its
          patches manage to pass all the test cases to which it is subjected.
\end{itemize}

Table~\ref{table:pass-rate-example} shows the results of our findings with the following
parameters: \texttt{n = 5}, \texttt{temperature = 0.8}, \texttt{filtered\_lab\_id = Lab-3}:

\begin{table}[h]
\centering
\begin{tabular}{|l|l|l|l|l|}
\hline
\textbf{Lab ID} & \textbf{Problem ID} & \textbf{\# Submission} & \textbf{\# Passed} & \textbf{Repair (\%)} \\ \hline
Lab-3           & 2810                & 16                     & 7                  & 43.75                \\ \hline
Lab-3           & 2811                & 11                     & 3                  & 27.27                \\ \hline
Lab-3           & 2812                & 19                     & 6                  & 31.58                \\ \hline
Lab-3           & 2813                & 17                     & 6                  & 35.29                \\ \hline
\textit{Total}  & \textit{--}         & \textit{63}            & \textit{22}        & \textit{34.92}       \\ \hline
\end{tabular}
\caption{Example of pass rate results.}

\label{table:pass-rate-example}
\end{table}

\subsubsection{Generating code completions given a problem description}

Our framework is also able to generate a variable number of completion solutions, given a natural
language prompt in the form of a problem description (see Appendix A for examples).
Similar to above, we are able to vary \texttt{n} (number of completions per problem]) and
\texttt{temperature}.
For our preliminary analyses, we have used the parameters \texttt{n = 5} and
\texttt{temperature = 0.8}.

\subsubsection{Devising a heuristic for measuring "syntactic distance" between two solutions}

Finally, we have managed to design a heuristic to compute the "syntactic distance" between two solutions
written in C.
As of now, we are only able to perform this heuristic on a pair of reference solution and its corresponding
(generated) patch solution.
Moreover, we only perform this heuristic if the patch solution can be compiled and is tested to be
semantically correct.
Analysing an incorrect patch solution in this step is simply meaningless and/or impossible.

The execution flow of this heuristic is as follows:
\begin{enumerate}
    \item We first parse each solution, written in C, using
          \href{https://github.com/eliben/pycparser}{pycparser, a third-party library written in Python
                to parse C code}.
    \item We then obtain each parsed solution as a custom \texttt{c\_ast.Node} object.
    \item For each \texttt{c\_ast.Node} object, we convert it into a \texttt{zss.Node} object.
          This is a necessary step as:
    \begin{itemize}
        \item It allows the converted object to now be processed in the final step of our heuristic.
        \item Each object will be assigned its corresponding label, which we use as the main source
              of "syntactic difference" across code solutions. (i.e. nodes with different labels
              will be considered to have a "syntactic distance" of \texttt{1})
    \end{itemize}
    \item Finally, to obtain the distance, we perform the algorithm to do so, designed by
          Zhang and Shasha~\cite{zhang1989simple}, and whose example implementations can be found
          \href{https://pythonhosted.org/zss/#module-zss}{here}.
\end{enumerate}
\chapter{Research Plan For Next Semester}
REPLACE WITH ACTUAL CONTENTS.

\section{Section 1}

\bibliographystyle{IEEEtranN}
\bibliography{report}

% Appendix (Remove if no appendix)
\appendix
\chapter{Dataset Used}
The following are the programming assignments used in testing the performance of our project,
including each problem's descriptions, as well as various (anonymous) student submissions and
model correct solutions.
%\input{contents/appendixb}

\end{document}
