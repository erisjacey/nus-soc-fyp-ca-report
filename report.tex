\documentclass[fyp,12pt]{socreport}

% Some generic packets.
\usepackage{color, colortbl}
\usepackage{url}
\usepackage{graphicx}
\usepackage{caption}
\usepackage{subcaption}
\usepackage{pgfplots}
\usepackage{tabularx}
\usepackage{multirow}
\usepackage{listings}
\usepackage{fullpage}
\pgfplotsset{width=10cm,compat=1.9}

% Sets the root path to look for all images.
\graphicspath{{images/}}

% Sets default options for listings.
\renewcommand\lstlistlistingname{List of Listings}
\newcommand*\lstinputpath[1]{\lstset{inputpath=#1}}
\lstinputpath{listings}
\lstset{frame=single, tabsize=2, captionpos=b}
\newcommand{\itab}[1]{\hspace{0em}\rlap{#1}}
\newcommand{\tab}[1]{\hspace{.11\textwidth}\rlap{#1}}

\begin{document}
\pagenumbering{roman}

% Replace as necessary
\title{Introductory Programming Assignment Repair via Program Transformations}
\author{Eris Jacey Masagca}
\projyear{AY 2022/2023}
\projnumber{H049520}
\advisor{Provost's Chair Prof Abhik Roychoudhury}
\deliverables{
    \item \itab{Report:} \tab{1 Volume}
    \item \itab{Manual:} \tab{1 Volume}
    \item \itab{Program:} \tab{1 Diskette}
    \item \itab{Database:} \tab{1 Diskette}
}
\maketitle

% Replace as necessary
\begin{abstract}
Test citation here\cite{ahmed2022verifix}.
The use of Wireless Sensor Networks for environmental monitoring has become
increasingly popular over the past decade due to its affordability, ease of deployment
and customisation, as well as its potentiality in the processing of sensed data. One of the
greatest challenges in this field would be in the design and implementation of an
efficient routing protocol which takes into account the various limitations of Wireless
Sensor Networks, such as battery life, limited storage capacities and high probability of
packet losses. Besides this, it is also extremely difficult to evaluate the performance of
such a protocol under crisis scenarios, due to its infrequency and unpredictability. In our
work, we have designed a routing protocol based on optimised Virtual Polar Coordinate
Routing (VPCR) (Newsome and Song, 2003) for use with our three-dimensional
testbed, comprising of 48 MICAz (Crossbow) motes spread across two floors of a
building. We have also developed a Java-based application with features for Event
Emulation and simple nodal analysis to assist us in our experiments. The overall
performance of our protocol will be gauged based on the average Path Stretch Factor
and path length comparisons between optimised and naïve VPCR.

\begin{descriptors}
    \item \itab{C.2.1}	\tab{Network Architecture and Design}
    \item \itab{C.2.2}	\tab{Network Protocols}
    \item \itab{C.2.4}	\tab{Distributed Systems}
    \item \itab{C.4} 	\tab{Performance of Systems}
    \item \itab{I.2.9}	\tab{Robotics}
\end{descriptors}
\begin{keywords}
    Software engineering, programming languages \& systems, automated program repair
\end{keywords}

% Replace/Delete as necessary
\begin{implement}
    \item{Ubuntu Linux 7.04 Feisty Fawn\\ TinyOS 2.x, NesC 1.2.8a, Java 1.6 SE, \\Xbow Motes}
    \item{Tembusu cluster}
\end{implement}

\end{abstract}

% Replace as necessary
\begin{acknowledgement}
See \texttt{README.md} of this repository.
\end{acknowledgement}

\listoffigures % Remove if no figures
\listoftables % Remove if no tables
\lstlistoflistings % Remove if no listings
\tableofcontents

% Actual contents in the contents folder.
% Include additional tex files here.
\chapter{Introduction}
This is the introduction of the example report using LaTex\cite{lamport1986document}.
Also an example citation.

\section{Goals}
This is some \textbf{BOLD} text.

This is some \textit{italicised} text.

This is some \texttt{monospaced} text.

This is some math: $a^2 + b^2 = c^2$.

\subsection{Contributions}
This is an example of an enumeration.

\begin{enumerate}
    \item Point one.
    \item Point two.
    \item Point three.
\end{enumerate}

\begin{lstlisting}[
    language=TeX,
    caption={Example Listing.}
]
\title{An Example HYP Final Report}
\author{Harold Finch}
\projyear{2015}
\projnumber{123-45-6789}
\advisor{Dr. Samaritan}
\deliverables{
    \item Report: 1 Volume
}
\maketitle
\end{lstlisting}
\include{contents/results}
\chapter{Progress Made So Far}
REPLACE WITH ACTUAL CONTENTS.

\section{Preliminary Design/Investigation}

\section{Design/Investigation Alternatives}
\chapter{Research Plan For Next Semester}
REPLACE WITH ACTUAL CONTENTS.

\section{Section 1}
\include{contents/conclusion}

\bibliographystyle{socreport}
\bibliography{report}

% Appendix (Remove if no appendix)
\appendix
\chapter{Dataset Used}
The following are the programming assignments used in testing the performance of our project,
including each problem's descriptions, as well as various (anonymous) student submissions and
model correct solutions.
\input{contents/appendixb}

\end{document}
