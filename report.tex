\documentclass[fyp,12pt]{socreport}

% Some generic packets.
\usepackage{color, colortbl}
\usepackage{url}
\usepackage{graphicx}
\usepackage{caption}
\usepackage{subcaption}
\usepackage{pgfplots}
\usepackage{tabularx}
\usepackage{multirow}
\usepackage{multicol}
\usepackage{listings}
\usepackage{fullpage}
\usepackage{hyperref}
\pgfplotsset{width=10cm,compat=1.9}

% Sets the root path to look for all images.
\graphicspath{{images/}}

% Sets default options for listings.
\renewcommand\lstlistlistingname{List of Listings}
\newcommand*\lstinputpath[1]{\lstset{inputpath=#1}}
\lstinputpath{listings}
\lstset{frame=single, tabsize=2, captionpos=b}
\newcommand{\itab}[1]{\hspace{0em}\rlap{#1}}
\newcommand{\tab}[1]{\hspace{.11\textwidth}\rlap{#1}}

\begin{document}
\pagenumbering{roman}

% Replace as necessary
\title{Introductory Programming Assignment Repair via Program Transformations}
\author{Eris Jacey Masagca}
\projyear{AY 2022/2023}
\projnumber{H049520}
\advisor{Provost's Chair Prof Abhik Roychoudhury}
\deliverables{
    \item \itab{Report:} \tab{1 Volume}
}
\maketitle

% Replace as necessary
\begin{abstract}

The use of automated feedback students attending introductory programming courses has become
increasingly popular due to its efficient and robust approach to feedback generation and error
detection, which can improve the grading performance of teachers and help students learn better
at the same time~\cite{haldeman2021automated}.
One of the greatest challenges in this field includes the accuracy of automated program repair (APR)
for incorrect student programming assignment submissions.
Existing techniques, such as TBar and Recorder, suffer from limitations like needing to rely on
a rich reference solution set for the same assignment.
In our work, we have been working on an intelligent tutoring system (ITS), which leverages and
improves on APR techniques.
The end-goal of this application is to aid students in CS-1 programming classes.
The overall performance of this project will initially be gauged based on various student submissions
and model correct solutions of programming assignments in one of the CS-1 classes at the National
University of Singapore (NUS): CS1010 Programming Methodology.

\begin{descriptors} % TODO
    \item \itab{TODO}	\tab{TODO}
%    \item \itab{C.2.1}	\tab{Network Architecture and Design}
%    \item \itab{C.2.2}	\tab{Network Protocols}
%    \item \itab{C.2.4}	\tab{Distributed Systems}
%    \item \itab{C.4} 	\tab{Performance of Systems}
%    \item \itab{I.2.9}	\tab{Robotics}
\end{descriptors}
\begin{keywords}
    Software engineering \\
    Programming languages \& systems \\
    Automated program repair
\end{keywords}

% Replace/Delete as necessary
\begin{implement}
    \item{Ubuntu 20.04.1 LTS, Windows 10}
    \item{Java 11 SE, IntelliJ IDEA 2020.2.3}
    \item{Python 3.8.5, PyCharm 2021.1.2}
\end{implement}

\end{abstract}

% Replace as necessary
%\begin{acknowledgement}
%See \texttt{README.md} of this repository.
%\end{acknowledgement}

%\listoffigures % Remove if no figures
%\listoftables % Remove if no tables
\lstlistoflistings % Remove if no listings
\tableofcontents

% Actual contents in the contents folder.
% Include additional tex files here.
\chapter{Introduction}

This project is being built on the backbone of the ITS we have been developing.
Some features of this system include providing an automated approach to APR,
providing students with tailored and dynamic feedback, as well as a robust auto-grader.

\section{Project Objectives}

One of the preliminary steps in the ITS framework requires the refactoring of reference correct
solutions in the hopes of more closely matching incorrect student programs.
As such, the first part of this project involves the implementation of refactoring rules as
outlined by an APR-centered paper~\cite{hu2019re}.
Subsequently, the second part revolves around investigating large language models (LLMs),
specifically Codex, and their efficacy in the APR process for CS-1 programming assignments.

% TEMPLATE BELOW %%%%%%%%%%%%%%%%%%%%%%%%%%%%%%%%%%%%%%%%%%%%%%%%%%%%%%%%%%%%%%%%%%%%%%%%%%%%%%%%%%%

%\section{Contributions}
%This is an example of an enumeration.
%
%\begin{enumerate}
%    \item Point one.
%    \item Point two.
%    \item Point three.
%\end{enumerate}
%
%\begin{lstlisting}[
%    language=TeX,
%    caption={Example Listing.}
%]
%\title{An Example HYP Final Report}
%\author{Harold Finch}
%\projyear{2015}
%\projnumber{123-45-6789}
%\advisor{Dr. Samaritan}
%\deliverables{
%    \item Report: 1 Volume
%}
%\maketitle
%\end{lstlisting}
\chapter{Literature Review}

%TO REVIEW (?):
%
%\begin{enumerate}
%    \item~\cite{ahmed2022verifix}
%    \item~\cite{chen2021evaluating}
%    \item~\cite{haldeman2021automated}
%    \item~\cite{hu2019re}
%    \item~\cite{zhang2022repairing}
%\end{enumerate}

\paragraph{Improving automatically generated code from Codex via Automated Program Repair}

\begin{itemize}
    \item Existing APR techniques like TBar and Recoder have their limitations in the context of
          fixing bugs in auto-generated programs.
    \item Codex-e has the potential of outperforming TBar and Recoder in code edit generation.
    \item The conventional way in which APR for introductory programming assignments still relies
          on referencing a rich reference solution set for the same assignment
    \item We propose:
    \begin{itemize}
        \item A matching mechanism to be able to search for similar past submissions from past
              assignments, which can then be used by language models.
        \item An incorrect fix guided refinement mechanism to further improve matching quality
              and repair success rate.
        \item An evaluation of the approach in real student submissions and showing of the value
              of historical assignments in APR
    \end{itemize}
\end{itemize}
\chapter{Progress Made So Far}
REPLACE WITH ACTUAL CONTENTS.

\section{Preliminary Design/Investigation}

\section{Design/Investigation Alternatives}
\chapter{Research Plan For Next Semester}
REPLACE WITH ACTUAL CONTENTS.

\section{Section 1}

\bibliographystyle{socreport}
\bibliography{report}

% Appendix (Remove if no appendix)
\appendix
\chapter{Dataset Used}
The following are the programming assignments used in testing the performance of our project,
including each problem's descriptions, as well as various (anonymous) student submissions and
model correct solutions.
%\input{contents/appendixb}

\end{document}
